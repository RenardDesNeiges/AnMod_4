\documentclass[11pt]{article}


\usepackage[margin=1in,a4paper]{geometry}
\usepackage[utf8]{inputenc}
\usepackage[T1]{fontenc}
\usepackage{lmodern}
\usepackage{tabularx}
\usepackage{quoting}
\usepackage{fancyhdr}
\usepackage{graphicx}
\usepackage{nicefrac}
\usepackage{sectsty}
\usepackage{graphicx}
\usepackage[T1]{fontenc}
\usepackage{epigraph} %quotes
\usepackage{amssymb} %math symbols
\usepackage{mathtools} %more math stuff
\usepackage{amsthm} %theorems, proofs and lemmas
\usepackage{optidef} %fast optimization problem notation
\usepackage{changepage}
\usepackage{gensymb}
\usepackage[ruled,vlined,noend,linesnumbered]{algorithm2e} %algoritms/pseudocode


\usepackage{biblatex}

%% asmthm notation
\newtheorem{theorem}{Theorem}[section]
\newtheorem{corollary}{Corollary}[theorem]
\newtheorem{lemma}[theorem]{Lemma}
\newtheorem{problem}{Problem}
\newtheorem{definition}{Definition}
\newtheorem{claim}{Claim}[section]


%% declaring abs so that it works nicely
\DeclarePairedDelimiter\abs{\lvert}{\rvert}%
\DeclarePairedDelimiter\norm{\lVert}{\rVert}%

\let\oldnl\nl% Store \nl in \oldnl
\newcommand{\nonl}{\renewcommand{\nl}{\let\nl\oldnl}}% Remove line number for one specific line in algorithm

\title{BIOENG-404 - Homework 4: SCONE}
\author{
    Titouan Renard
    - MT : Robotics 
}
\date{\today}



\begin{document}


\maketitle

\section{Healthy Gait}

The solution provided by the optimization process (which converges relatively fast) is quite good. The gait appears natural when played as a video but a close examination of the gait parameters (pelvic tilt, hip, knee and ankle flexion as well as ground reaction forces, displayed in figure \ref{healthy_gait}) show that the model gives results slightly off compared to measured data in humans. The curves have the approximate shape of the real-life data but their amplitude and timing is slightly off. Furthermore one observes that the ankle flexion displays a dip in angle at $\sim 25\%$ of the gait cycle that simply doesn't exist in humans. This model seems more useful as a toy-model, to rapidly test out hypotheses rather than as an accurate prediction tool (as is often the case when modeling complex dynamical systems). The render allows for the observation of gait kinematics as displayed in figure \ref{healthy_render}.

\begin{figure}[h!]
    \centering
    \includegraphics[width=0.9\textwidth]{screens/healthy_gait.png}
    \caption{Gait parameters as observed in the SCONE simulation of the gait of a healthy subject. The grayed-out area show distribution of gait parameters in healthy humans, the colored lines show the gait parameter evolution over multiple gait cycle for our optimal solution.}
    \label{healthy_gait}
\end{figure}

\begin{figure}[h!]
    \centering
    \includegraphics[width=\textwidth]{screens/healthy_render.jpg}
    \caption{Rendering of the optimization solution produced by the healthy model and controller. The stills are presented in chronological order from left to right. T red line denotes a reference point in space between the different stills. One can clearly observe heel contact (on the leftmost still), a foot-flat phase (on the two centered stills) and then toe-off (on the rightmost still).}
    \label{healthy_render}
\end{figure}

\section{Pathological Gait : Plantarflexor Muscle Atrophy}

We reduce the maximum force that can be produced by the plantarflexor muscles (\textit{gastrocnemius} and \textit{soleus}). The principal kinetic adaptation is (quite obviously) a reduction of the amplitude of plantarflexion as is clearly visible on the ankle figure plot of the gait analysis in figure \ref{atrophic_gait}. A more general observation is that the gait analysis signal is generally more irregular as it seems this system is harder to control by the control architecture in our model. The speed of walking is also slower (we relaxed the speed constraint before launching the optimization process).

\begin{figure}[h!]
    \centering
    \includegraphics[width=\textwidth]{screens/atrophy_gait.png}
    \caption{Gait parameters as observed in the SCONE simulation of a subject with a gait affected by PF muscular atrophy.}
    \label{atrophic_gait}
\end{figure}

The resulting gait displays a \textit{heel-walking} behavior as is clearly visible on the rendering of figure \ref{atrophic_render}. As the body cannot produce any torque on the heel the lays flat on the ground for the entire support phase.

\begin{figure}[h!]
    \centering
    \includegraphics[width=\textwidth]{screens/heel_walk.jpg}
    \caption{Rendering of the optimization solution produced for the PF muscular atrophy model.}
    \label{atrophic_render}
\end{figure}

Compared with the healthy biomechanics model, it takes much longer for the muscle-atrophied model to converge to a stable gait ($\sim 80$ iterations instead of $\sim 10$ iterations) it also converges to a higher cost ($0.797$ is the best score we obtained for the healthy gait, when modeling muscle atrophy we only converge to $0.989$, where cost is a metric that estimates metabolic cost of walking and penalizes falling, injury). It appears the muscle atrophy gait is necessarily less energy efficient than the healthy gait.

\section{Pathological Gait : Hyperreflexia}

We increase the gain of velocity dependent reflexes for both plantarflexor muscles in the model controller and run the optimization process until we find a stable gait. We clearly see that the principal kinematic adaptation is that the model adopts a \textit{toe-walking} gait. This can be observed in the ankle flexion plot  from the gait analysis (figure \ref{hr_gait}) which adopts a constant negative value (the heel is almost never flat on the ground) or in the renders (figure \ref{hr_render}) that clearly show that the body supports itself on the toe alone for the entire support phase. The knee flexion amplitude is also increased. The pelvic tilt also displays a higher average value than for both previous gaits (which actually makes it more in line with average, healthy human walking).

\begin{figure}[h!]
    \centering
    \includegraphics[width=\textwidth]{screens/hyperreflexia_gait.png}
    \caption{Gait parameters as observed in the SCONE simulation of a subject with a gait affected by hyperreflexia. We observe that the ground reaction force displays high frequency oscillations, the signal has a higher amplitude on every variable. }
    \label{hr_gait}
\end{figure}

Furthermore we observe that the gait signal is significantly more irregular (less stable) on all variables compared with both the healthy and muscle atrophic gait. From a control theory perspective it actually quite unsurprising that a stronger feedback leads to a higher instability in a closed loop dynamical system.

\begin{figure}[h!]
    \centering
    \includegraphics[width=\textwidth]{screens/toe_walk_hr.jpg}
    \caption{Rendering of the optimization solution produced for the hyperreflexia model.}
    \label{hr_render}
\end{figure}

Perhaps surprisingly, the hyperreflexia model converges to a stable gait about as fast as the healthy model ($\sim 12$ iterations for the hyperreflexia model v.s. $\sim 10$ iterations for the healthy model), but the cost function converges to a higher value ($1.673$ for hyperreflexia compared with $0.797$ for the healthy gait). The fast convergence might be attributed to the more irregular dynamics that are associated with a higher feedback gain (the optimization algorithm might get out of local minima faster). The higher final cost is probably also attributed to instability (cost function penalizes head movement, falls and high energy use, which are all associated with high-amplitude, high frequency muscle activation).

\section{Pathological Gait : Toe Walking}

We try to reproduce a pathological gait that displays toe-walking. To get to such a gait we propose an alteration to the OpenSim model to emulate plantarflexor muscle contracture (as proposed by Ong et al. \cite{1}). We implement severe PF contracture by reducing the optimal fiber length of left and right \textit{soleus} an \textit{gastrocnemius} muscles by a factor of $0.7$. After running the optimization process we observe that the solution indeed converges to a toe-walking gait that is quite similar to the one observed when modeling hyperreflexia. The resulting gait displays less high frequency oscillations than the one obtained by modeling hyperreflexia (figure \ref{toe_gait}), but the gait is quite irregular (it differs from one cycle to another). The knee flexion amplitude is also lower than in the hyperreflexia example.

\begin{figure}[h!]
    \centering
    \includegraphics[width=\textwidth]{screens/toe_walk_gait.png}
    \caption{Gait parameters as observed in the SCONE simulation of a subject with a gait affected by severe PF contracture.}
    \label{toe_gait}
\end{figure}

The biomechanical intuition behind the emergence of toe-walking when modeling severe PF contracture is actually quite simple : shortening the plantarflexor muscles means that the rest position of the ankle is no longer flat on the ground. Therefore the most energy efficient walking cycle becomes one where ankles are flexed in the direction of plantarflexion, i.e. toe-walking.

\begin{figure}[h!]
    \centering
    \includegraphics[width=\textwidth]{screens/toe_render.jpg}
    \caption{Rendering of the optimization solution produced for the toe-walking model (subject affected with severe PF contracture).}
    \label{toe_render}
\end{figure}

The PF contracture model converges fast to a stable gait ($\sim 10$ iterations, about the same as the healthy model) but the terminal cost is really high ($1.182$ which is lower than the hyperreflexia solution but higher than the muscle atrophy solution). This is most likely explained by the same factors as in the hyperreflexia example : irregular walking, muscle activation higher on average.


\begin{thebibliography}{1}

    \bibitem{Ong} Ong CF., Geijtenbeek T., Hicks JL., Delp SL., "\textit{Predicting gait adaptations due to ankle plantarflexor muscle weakness and contracture using physics-based musculoskeletal simulations}" \emph{PLoS Computational Biology 15}, 2019.
\end{thebibliography}


\end{document}

